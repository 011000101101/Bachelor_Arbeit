%!TEX root = ../Thesis.tex

\begin{otherlanguage}{ngerman}

%*******************************************************
\chapterExtra{Erklärung zur Dissertationsschrift}
%*******************************************************

% Promotionsordnung und mehr des FB 20
% https://www.informatik.tu-darmstadt.de/forschung_fb20/wissenschaftliche_karriere/promotion/index.de.jsp

\begin{flushright}
    \emph{\small gemäß §\,9 der Allgemeinen Bestimmungen der Promotionsordnung der \\
    \myUni{} vom \formatdate{12}{1}{1990} (ABI. 1990, S.\,658) \\
    in der Fassung der 8.\,Novelle vom \formatdate{1}{3}{2018}}
\end{flushright}
Hiermit versichere ich, \myName{}, die vorliegende Dissertationsschrift ohne Hilfe Dritter und nur mit den angegebenen Quellen und Hilfsmitteln angefertigt zu haben. Alle Stellen, die Quellen entnommen wurden, sind als solche kenntlich gemacht worden. Eigenzitate aus vorausgehenden wissenschaftlichen Veröffentlichungen werden in Anlehnung an die Hinweise des Promotionsausschusses \myFacultyDE{} zum Thema \textquote{Eigenzitate in wissenschaftlichen Arbeiten} (EZ-2014/10) in Kapitel \textquote{\emph{Previously Published Material}} auf \cpagerefrange*{ch:PreviousPublications}{ch:PreviousPublicationsEnd} gelistet. Diese Arbeit hat in gleicher oder ähnlicher Form noch keiner Prüfungsbehörde vorgelegen. In der abgegebenen Dissertationsschrift stimmen die schriftliche und die elektronische Fassung überein.

\bigskip

\noindent\textit{\myLocation{}, \myTime{}}

\begin{flushright}
    \begin{tabular}{m{5cm}}
        \\ \hline
        \centering\myName{} \\
    \end{tabular}
\end{flushright}

\end{otherlanguage}
