% !TeX root = ../Thesis.tex

%*******************************************************
% Abstract
%*******************************************************
%\renewcommand{\abstractname}{Abstract}
\pdfbookmark[1]{Abstract}{Abstract}
% \addcontentsline{toc}{chapter}{\tocEntry{Abstract}}
\begingroup
\let\clearpage\relax
\let\cleardoublepage\relax
\let\cleardoublepage\relax

\chapter*{Abstract}

Today, advanced machine learning is deployed nearly everywhere, often to great success. Yet, the field of hardware high-level synthesis has experienced notably little of this recent trend. 

Deploying artificial neural networks in the Placement stage of the synthesis flow shows great potential for improvement over traditional approaches. Using neural network based wirelength estimation can enable high-quality placements at reduced runtimes on some circuits. Increased research into this application of deep neural networks is likely to yield generally applicable solutions, innovating the design flows established in the field of hardware synthesis.

\vfill

\begin{otherlanguage}{ngerman}
\pdfbookmark[1]{Zusammenfassung}{Zusammenfassung}
\chapter*{Zusammenfassung}

Fortgeschrittene Techniken des Maschinellen Lernens werden heutzutage quasi überall eingesetzt, oft mit großem Erfolg. Das Gebiet der Hardwaresynthese wurde jedoch merklich wenig von diesem aktuellen Trend beeinflusst.

Künstliche neuronale Netze in der Platzierungsphase des Synthesevorgangs einzusetzen zeigt großes Potenzial für Verbesserungen gegenüber traditionellen Ansätzen. Der Einsatz einer Verdrahtungslängenabschätzung basierend auf neuronalen Netzen kann bei gewissen Schaltungen qualitativ hochwertige Platzierungen bei verminderter Laufzeit ermöglichen. Verstärkte Forschung an diesem Einsatz von tiefen neuronalen Netzen wird mit hoher Wahrscheinlichkeit weitreichend anwendbare Lösungen hervorbringen, was die etablierten Designprozesse im Gebiet der Hardwaresynthese weiter verbessern würde.

\end{otherlanguage}

\endgroup

\vfill
