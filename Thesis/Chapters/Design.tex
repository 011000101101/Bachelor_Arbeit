% !TeX root = ../Thesis.tex

%************************************************
\chapter{Design}\label{ch:design}
%************************************************
\glsresetall % Resets all acronyms to not used

Adding \glspl{NN} to \gls{VPR} consists of two parts: Designing and training the \glspl{NN}, and modifying the \gls{VPR} Placer to call them instead of the usual \gls{HPWL} routine.

\section{Problem Definition}

The problem to be solved with \glspl{NN}, that \gls{VPR} usually solves with \gls{HPWL}, is wirelength estimation.

The input is the temporary placement of a net, encoded as a list of 2D grid coordinates, with arbitrary length, but at least two entries. Each 2D coordinate is a pair of non-negative integers denoting the X and Y position of the terminal in the grid. 

The first element of the list specifies the position of a signal source, followed by the positions of all sinks connected to this source in the order in which they are stored in \gls{VPR}.

The output is the wiring cost for wiring this net in isolation using the Maze-Router. This is the minimum wire length achievable when greedily routing one terminal at a time in a fixed order.

\section{Neural Networks}

This thesis compares two different types of \glspl{NN} which are both suited to the problem at hand, namely \glspl{CNN} and \glspl{RNN}.

\subsection{\gls{CNN}}

Due to the discrete nature of possible placement positions on a \gls{FPGA}, every distinct pair of terminal coordinates represents a position on the equally spaced placement grid. 

As such, the grid itself can be represented by a binary-valued image (given a fixed grid size), where the pixels represent possible placement positions, and each value shows if the respective position contains a terminal of the current Net or if it is empty.

\subsubsection{Pros}

The main advantage of this approach is that every possible input can be converted to a tensor of fixed size, if a maximum grid size is specified, and each image represents this maximum grid. Therefore, samples converted in this way could already be processed by a single \gls{CNN} without further preprocessing, although quite inefficiently.

\subsubsection{Cons}

The main disadvantages are the memory-inefficient encoding, the loss of ordering information, and the limitation to a maximum grid size.

The encoding is an equivalent of a Bitmap, which is the most inefficient common image encoding in terms of memory requirement, as it contains no compression and has a fixed size given fixed dimensions.

In order to reduce the memory overhead of this approach, it is adapted to relative placement positions inside the bounding box of each Net, instead of absolute positions on the \gls{FPGA} grid. This reduces the memory requirement of each sample from the maximum placement grid size to the maximum bounding box size of all considered Net placements.

Furthermore, the ordering of the terminals inside a Net is lost. While the minimum possible wire length of a list of terminals is ordering invariant, the actual routing in \gls{VPR} is performed using the Maze-router, an ordering sensitive heuristic. Therefore, different internal orderings of the same list of terminal positions might produce routings of differing quality, and subsequently should be rated accordingly by the wire length estimation heuristic.

Our assumption at this point is that the influence of the ordering is small enough, that \glspl{CNN} are still able to perform better than \gls{HPWL} on the accuracy/runtime trade-off.

Finally, this approach is limited to a maximum grid, or rather Net placement bounding box, size. Any fixed size encoding can by design only represent problem instances up to a certain size. Therefore, any instance exceeding this limit can not be processed a \gls{CNN}. Furthermore, as the maximum size (number of grid positions) linearly influences the computational effort for each prediction, it is impractical to simply set the limit so high an excess becomes highly unlikely.

Therefore, to be able to process any valid input, a fallback estimator for problem instances exceeding the limit is required. While an ensemble of \glspl{CNN} with increasing input dimensions could combat the computational effort for prediction, it would require lots of resources for training, and still defines an upper limit that might be exceeded. Thus, a single \gls{CNN} able to process all commonly encountered Net placements on the \gls{VPR} benchmarks was chosen, and any placement exceeding this limit will fall back to \gls{HPWL}.

\subsection{\gls{RNN}}

In \gls{VPR}, the placement of the Terminals in a Net is represented by an ordered list of coordinates. The length of this list equals the number of terminals in the Net and is, in theory, unbounded. Due to the ordering sensitivity of the problem, and the variable-length input sequences, \glspl{RNN} are inherently suited to this problem.

This approach looks promising, as the computational structure of \glspl{RNN} mimics the process of sequentially adding a new terminal to the partial routing used in the Maze-router, which is the target function the \gls{NN} is trying to approximate.

The computational effort of this approach scales linearly with the number of terminals in a Net but is expected to be lower than the quasi-constant effort of \glspl{CNN}, as the number of terminals is generally lower than the number of distinct placement positions in the Nets \gls{BB}, by the maximum of which the effort of our CNN approach is scaled.

\section{Integration}

Each of the \glspl{NN}, wrapped in an interface, directly replaces the \gls{HPWL} heuristic used in the original \gls{VPR} implementation.

The interface pre-processes the input (the terminal coordinates of a net), executes the \gls{NN} inference, and post-processes the result to have the same unit as the cost computed by \gls{HPWL}.

However, parts of the \gls{HPWL} implementation, namely \gls{BB} computation and updating, are kept to map and normalize the input.

Thus, the actual cost computation part of the \gls{HPWL} implementation with constant runtime is replaced with \gls{NN} inference, which generally has substantially higher runtime. However, it is expected to also yield substantially more accurate wire length estimations, hopefully improving the placement quality enough to make up for the increased computational effort.


